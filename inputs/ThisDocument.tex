\section{This document}

The subject of this dissertation is a symmetry, that, if realized in the context of the $\sm$, may resolve a number of the above mentioned riddles, including the hierarchy problem. The symmetry, known as $\SUSY$ \cite{Martin:1997ns}, is the only remaining spacetime symmetry not yet found in nature. Supersymmetry offers a potentially viable explanation for the existence of dark matter, and if ever discovered, could motivate certain models of grand unification, theories of everything, and string theories. 

This document summarizes work that has been carried out to shed light on the question of whether a supersymmetric extension of the $\sm$ is possible given the observations made so far by the Compact Muon Solenoid (CMS), and if not, whether a definitive answer can be expected from analysis of current and future experimental data. Additionally, analysis techniques are discussed that have been applied to the Run 2 (2015) dataset. I have authored the text in a style suitable for a graduate student interested in contributing to the effort of globally understanding the experimental viability of supersymmetry, and in developing new techniques to search for  supersymmetry at the Large Hadron Collider (LHC).

The body of this document is organized as follows. I proceed with a brief description of supersymmetry and how it may manifest itself at the LHC.  Chapter \ref{chap:lhc} gives a description of the LHC, and Chapter \ref{chap:cms} provides a description the Compact Muon Solenoid (CMS). During the discussion of the CMS detector, I make short forays into studies and techniques I have developed while contributing to a number of technical upgrades within CMS. 

In Chapter \ref{chap:run1pmssm}, I describe an analysis~\cite{bib:Me!} I helped perform within CMS to understand how the hypothesis of supersymmtry is constrained by the data that has been collected and analyzed by CMS and other experiments. The analysis is a global interpretation of a representative set of CMS SUSY searches in the context of a general model of supersymmetry that is a proxy for the 120-parameter model called the minimal supersymmetric standard model. The submodel is called the phenomenological MSSM (pMSSM). The goal of the chapter is to summarize what lessons have been learned about the viability of supersymmetric hypotheses based on LHC results. 

In Chapter \ref{chap:susysearches}, I proceed to describe various analysis techniques that I have developed to improve the sensitivity of CMS searches to supersymmetry models that have not been excluded by the CMS Run 1 (2011-2012) analyses. This includes novel methods for modeling of Standard Model QCD and $\zinv$ processes, which are dominant sources of background in searches that look for evidence of supersymmetry, as well as novel techniques for modeling triggers (triggers are defined in Chapter \ref{chap:cms}) useful for analyses that are brave enough to hunt for supersymmetry in kinematic regions where the background is dominant. I also give key details about two 2015 and 2016 analyses that I contributed to while developing the mentioned techniques \cite{Khachatryan:2016kdk}\cite{CMS:2016nhb}. The chapter ends with a proof of principal demonstration of a bold strategy for searching for evidence of supersymmetry, namely, the feasibility of employing multivariate discriminants to maximize sensitivity to models of supersymmetry.

Chapter \ref{chap:ma5} summarizes contributions I made to a project~\cite{Dumont:2014tja} outside the CMS collaboration that has the goal of making the results of CMS and ATLAS SUSY searches more accessible to the theoretical physics community. 
