\chapter{Introduction}

A series of discoveries spanning three centuries has brought to light the existence of 25 elementary particles. These particles, along with the forces that govern their interactions, are described by a mathematical model many consider to be one of the great intellectual achievements of the 20th century. The model, called the $\SM$ of particle physics (for a thorough review, see Ref. \cite{Rosner:2001zy}), has enabled scientists to understand the process by which the sun radiates light, the forces that bind together solid matter, and details about the history of the universe extending back nearly to the beginning of time. The $\sm$ has been tested by countless observations from hundreds of astronomical and collider experiments, and its predictions have been borne out almost without exception. But despite its extraordinary success, the $\sm$ is not believed to give a complete picture of nature.

There are three reasons the $\sm$ is thought to be incomplete. The first arises from a handful of laboratory and astronomical findings that are inconsistent with predictions. These findings include the discovery of neutrino masses \cite{Fukuda:1998mi}, as well as the measured magnetic dipole moment of the muon \cite{Bennett:2006fi}. The second is the observation of so-called emergent phenomena, behaviors of systems that become manifest as the number of particles making up the systems grows. {\it More is Different} \cite{Broglia:2012ef} describes several examples of emergent phenomena, including the appearance of high-temperature superconductivity, as well as the quantum Hall effect, suggesting that the reductionist $\sm$ is of limited predictive power in large systems. The third reason the $\sm$ is believed to be incomplete is the presence of fine tuning; that is, patterns that cannot not be explained without the addition of some new physics beyond the $\sm$ (BSM). These patterns, which are embedded in the $\sm$, perhaps hint at the existence of some new structure,  like supersymmetry, or at the existence of many other universes. The latter explanation is sometimes viewed as non-scientific because it doesn't appear to make testable predictions, but the former, as will be seen, is well-motivated, and can be tested by studying the collisions of highly energetic particles. To illuminate these arguments, it is necessary to first explain what exactly the $\sm$ is.

The $\sm$ is an example of a general class of theories called \qfts. The task of Chapter \ref{chap:QFT} of this dissertation is to define \qft, providing an amount of detail needed to define and contextualize the nomenclature used throughout the dissertation, and to provide enough detail as is necessary to then describe the $\sm$ in Chapter \ref{chap:SM}. A reader who is already familiar with QFT and the Standard Model may choose to skip these chapters. Following the description of the $\sm$ is a discussion of supersymmetry (SUSY) that focuses on how SUSY might manifest itself in the context of the $\sm$. Then, sections describing two experimental facilities, the CERN Large Hadron Collider (LHC) and the Compact Muon Solenoid (CMS) detector, describe how these facilities operate and why they are ideal for testing most variants of the hypothesis of $\SUSY$. Embedded within the CMS chapter are subsections detailing work that has been carried out to examine, improve and maintain the performance of particle detectors. 

The remainder of the dissertation consists of sections describing the results of work carried out in search of evidence of supersymmetric phenomena in the particle collisions observed and recorded by the CMS detector during the 2015 run of the LHC (Run 2). These sections include: a detailed analysis quantifying how searches for supersymmetry at CMS have constrained our knowledge of $\SUSY$; a discussion of data analysis techniques that increase the sensitivity of BSM searches performed at CMS, including developments in the modeling of Standard Model background events arising from quantum chromodynamics (QCD) interactions as well as interactions that produce electroweak bosons; and a demonstration of new methods that may allow LHC searches to detect certain models of supersymmetry using subsets of data that are not traditionally considered for analysis. Finally, a summary is given on work that has been performed outside the CMS collaboration to create and support a public database of LHC BSM physics analyses. The database contains publicly-available implementations of CMS analyses that can be used by independent theorists to constrain BSM models.