%%%%%%%%%%%%%%%%%%%%%%%%%%%%%%%%%%%%%%%%%%%%%%%%%%
\section{Standard Model background predictions}
%%%%%%%%%%%%%%%%%%%%%%%%%%%%%%%%%%%%%%%%%%%%%%%%%%

%-------------
The event selection described in Sec.~\ref{sec:pre-selection} removes most of the SM background, including \ttbar and $W$+jets, but residual contributions remain due to the following mechanisms. 
%
If one or both $W$'s decay leptonically and the leptons are not reconstructed or identified, the result would be an event with all jets and \met that would mimic a signal event. 
%
Similarly, if one or both $W$'s decayed leptonically to $\tau\nu$ and the $\tau$'s decayed hadronically, we would also have an all-hadronic final state with \met.  
%
Another important background for any search involving jets and large $\ptvecmiss$ is the production of $\cPZ$ bosons in association with jets, especially b-jets, where the $\cPZ$ boson decays to neutrinos.  
%
Further, in the event of a catastrophic momentum mismeasurement of one or more jets in a SM multijet event, large amounts of spurious \MET are present in the reconstructed event, which can potentially mimic an all-hadronic SUSY final state that passes the search selection.   
%
Besides the dominant backgrounds described above, other SM background processes with lower cross sections can mimic the signal, such as diboson ($WW$, $WZ$, $ZZ$) processes, multiboson ($WWW$, $WWZ$, $ZZZ$) processes, and the associated production of a $W$ or $Z$ boson with a top quark pair (\ttbarW and \ttbarZ). 
%
Due to the reconstructed top requirement, the \ttbarW and \ttbarZ processes are expected to have larger contributions than the rest. 


