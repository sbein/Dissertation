%%%%%%%%%%%%%%%%%%%%%%%%%%%%%%%%%%%%%%%%%%%%%%%%%%
\subsection{\texorpdfstring{Backgrounds from \ttbarZ and other SM rare processes}%%%%%%%%%%%%%%%%%%%%%%%%%%%%%%%%%%%%%%%%%%%%%%%%%%
{Backgrounds from ttZ and other SM rare processes}}
\label{sec:ttZ}
%%%%%%%%%%%%%%%%%%%%%%%%%%%%%%%%%%%%%%%%%%%%%%%%%%

%-------------
Similar to the $Z\rightarrow\nu\nu$+jets background, \ttbarZ is an irreducible background when $Z$ bosons decay to $\nu \nu$ and both top quarks decay hadronically. 
%
The \ttbarZ cross section at 13~TeV is 782.6~\pb, so the predicted yield of \ttbarZ events in the search bins is less than 10\% of the total background. 
%
Given the small cross section associated with this process, we rely on simulation to predict the contribution of \ttbarZ events to each search region bin, although this estimation is validated using data.
%
The \ttbarW background estimation is covered by the lost lepton and hadronic tau background methods. 

%-------------
%The dedicated cross section measurement of \ttbarZ and \ttbarW published in Ref.~\cite{Khachatryan:2015sha} at 8TeV, demonstrates that the three-lepton channel, where the $Z$ boson decays into two electrons or two muons and one $W$ boson decays leptonically while the other $W$ boson decays hadronically, has the best signal to backgrounds ratio. 
%
%Consequently, we study the \ttbarZ process in the three-lepton channel.

%-------------
In the data validation of the \ttbarZ simulation, we start from a single muon triggered data sample, corresponding to an integrated luminosity of 2.3\ifb. 
%
Events are selected to maximize the presence of \ttbarZ events and minimize its background. 
%
They are required to have at least 4 jets with the same \pt requirements as for the pre-selection, two muons and an additional electron or muon, and at least one $b$-tagged jet to suppress Drell-Yan background.  
%
The leading muon must have a \pt be greater than 45 \gev and the second leading muon must $\pt>$20 \gev. 
%
These thresholds are chosen to ensure that the muons are in the plateau range of the trigger efficiency curve.  
%
A third lepton is required and must have a \pt greater than 10 \gev. 
%
At least one muon pair must have an invariant mass in the 71-111 \gev range. 
%
This tight mass window has been optimized to suppress \ttbar
background from the \ttbarZ sample. 

%-------------
The MC simulations are normalized to their corresponding cross sections. 
%
In the differential \ttbar cross section measurements~\cite{CMS-PAS-TOP-12-027,CMS-PAS-TOP-12-028}, the shape of the \pt spectrum of the individual top quarks in data is found to be softer than predicted from simulation. 
%
Therefore, \ttbar MC simulated samples are reweighted to account for this effect.  
%
After subtracting the contributions from backgrounds, 6.4 events in total, an estimated \ttbarZ contribution of 5.6 events is measured, which is only a factor of 1.167 greater that the pure MC simulation prediction. 
%
The statistical uncertainty in the data measurement is about 30\%. 

The yields for diboson and multiboson processes are combined into singla rare background prediction, which are fully determined by simulation. 
%
Their predictions, in addition to the \ttbarZ prediction, are showed in Appendix~\ref{sec:bkgpred} in Table.~\ref{tab:Rarepred}.
